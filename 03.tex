\chapter{目的と手法}
\label{chap:design}

この章では、本研究の目的を明らかにした後、どのような手法を用いるかを示す。

\section{目的}

本研究の目的は、一定のルールでソースコードを日本語に変換することによって可読性を高め、プログラムを理解するコストを下げることができるかどうかを検証することである。今回はプログラミング言語はpythonを選んだ。pythonは簡潔に記述できることがメリットのプログラミング言語であると同時に、変数宣言と代入の区別がないことやインデントでのブロック管理、型がないなど可読性という面ではほかの言語よりも劣っている面が多い言語である。つまりpythonで効果を実証できればほかの言語でも、一定の効果が得られる可能性は高い。よって本論文ではpythonを例に日本語に変換する機能を実装しいくつかの例題で評価をとり、実際にソースコードの理解を補助することができるのかを検証する。

\section{手法}

検証するための手法として、まずpythonのコードを機械的に日本語に翻訳する。そして出力された日本語と、同じ処理をするプログラミング言語「ひまわり」のコードと比較してアンケートを取りその結果を検証結果とする。

\section{日本語変換}

ソースコードから日本語に変換する処理の流れとしてはpythonの抽象構文木を日本語にしてく