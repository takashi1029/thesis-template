\chapter{関連研究}
\label{chap:related}


\section{テンプレートの構成}

このテンプレートは、表\ref{tb:files}のファイルで構成されている。

\begin{table}[htbp]
  \caption{構成ファイル}
  \label{tb:files}
  \begin{center}\begin{tabular}{c|l}
    \hline
    ファイル名&用途\\\hline\hline
    {\tt main.tex}&メインのファイル。これを編集していく\\\hline
    {\tt thesis.sty}&論文のスタイルを定義したファイル。基本的には手は加えない\\\hline
    {\tt *.tex}&{\tt main.tex}に{\tt include}されるファイル群\\\hline
    {\tt *.eps}&画像ファイル\\\hline
    {\tt main.bib}&参考文献用のBibTeXファイル\\\hline
    {\tt Makefile}&Makefile。次節以降で説明\\\hline
    {\tt .gitignore}&Git用設定ファイル\\\hline
  \end{tabular}\end{center}
\end{table}

\section{コンパイル}
このテンプレートの\LaTeX ファイルをコンパイルしてPDFファイルを生成するには、ターミナルを開いて以下のようにする。

\begin{itembox}[l]{コマンド実行例}
\begin{verbatim}
% make
\end{verbatim}
\end{itembox}

こうすることで、\verb|platex|コマンド、\verb|pbibtex|コマンド、\verb|platex|コマンド2回、\verb|dvipdfmx|コマンドが全て実行され、{\tt main.pdf}が生成される。

コンパイルによって生成されたファイルを全て消すには、以下のようにする。

\begin{itembox}[l]{コマンド実行例}
\begin{verbatim}
% make clean
\end{verbatim}
\end{itembox}

\section{設定}

以下、{\tt main.tex}に対して行うべき設定を、このファイルの中に書いてある順に沿って説明する。

\subsection{論文全体の言語の設定}
\label{sec:lang}

\begin{itembox}[l]{{\tt main.tex}}
\begin{verbatim}
\japanesetrue	% 論文全体を日本語で書く(英語で書くならコメントアウト)
\end{verbatim}
\end{itembox}

ここでは論文全体の言語を設定する。日本語に設定すれば、『章』『目次』『謝辞』などが日本語で出力されて、行頭のインデントなども日本語の仕様になる。英語にした場合は、これらはそれぞれ『Chapter』『Table of Contents』『Acknowledgment』な体裁になる。インデントも行間も、英語用の設定が適用される。

\verb|\japanesetrue| をコメントアウトしなければ日本語に、コメントアウトすれば英語に設定される。


\subsection{余白の設定}

\begin{itembox}[l]{{\tt main.tex}}
\begin{verbatim}
\bindermode	% バインダ用余白設定
\end{verbatim}
\end{itembox}

このテンプレートの出力はA4用紙。ここではこれの四辺の余白を設定する。

最終的にバインダーで綴じて提出する場合、余白を左右対称にしてしまうと、見かけ上のバランスがとても悪くなる。これを解消するため、あらかじめ左側の余白を大きく取っておく。

\verb|\bindermode| をコメントアウトしなければ左綴じ用の余白に、コメントアウトすれば左右対称の余白に設定される。

両面印刷の場合、偶数ページと奇数ページで余白を広くとるべき側が違うので、\verb|documentclass| でこれを設定する。

\begin{itembox}[l]{{\tt main.tex}}
\begin{verbatim}
% 両面印刷の場合。余白を綴じ側に作って右起こし。
\documentclass[a4j,twoside,openright,11pt]{jreport}
% 片面印刷の場合。
%\documentclass[a4j,11pt]{jreport}
\end{verbatim}
\end{itembox}

両面印刷の場合は \verb|twoside| を使用する。\verb|openright| を使うと章のはじまりが必ず右側のページに来るようになる。

\subsection{論文情報の設定}
\label{sec:meta}

\begin{itembox}[l]{{\tt main.tex}}
\begin{verbatim}
% 日本語情報(必要なら)
\jclass  {修士論文}                             % 論文種別
\jtitle    {修士論文用 \LaTeX\ テンプレート}    % タイトル。改行する場合は\\を入れる
\juniv    {慶應義塾大学大学院}                  % 大学名
\jfaculty  {政策・メディア研究科}               % 学部、学科
\jauthor  {ほげ山 ふう助}                       % 著者
\jhyear  {24}                                   % 平成○年度
\jsyear  {2012}                                 % 西暦○年度
\jkeyword  {\LaTeX、テンプレート、修士論文}     % 論文のキーワード
\jproject{インタラクションデザインプロジェクト} %プロジェクト名
\jdate{2013年1月}

% 英語情報(必要なら)
\eclass  {Master's Thesis}                            % 論文種別
\etitle    {A \LaTeX Template for Master Thesis}      % タイトル。改行する場合は\\を入れる
\euniv  {Keio University}                             % 大学名
\efaculty  {Graduate School of Media and Governance}  % 学部、学科
\eauthor  {Fusuke Hogeyama}                           % 著者
\eyear  {2012}                                        % 西暦○年度
\ekeyword  {\LaTeX, Templete, Master Thesis}          % 論文のキーワード
\eproject{Interaction Design Project}                 %プロジェクト名
\edate{January 2013}
\end{verbatim}
\end{itembox}

ここでは論文のタイトルや著者の氏名などのメタデータを記述する。ここで書いたデータは、表紙とアブストラクトのページに使われる。必ずしも日本語と英語の両方を設定しなければいけないわけではなくて、自分が必要とする方だけ記述すればよい。

タイトルが長過ぎる場合は、表紙やアブストラクトのページでは自動で折り返して出力される。もし改行位置を自分で指定したい場合は、その場所に \verb|\\| を入力する。


\section{出力}

\verb|\begin{document}| から \verb|\end{document}| に記述した部分が、実際に{\tt DVI}(最終的には{\tt PDF})ファイルとして出力される。

\subsection{外部ファイルの読み込み({\tt include})}

出力部分の具体的な説明の前に、外部ファイルを読み込む方法を説明する。

\verb|\begin{document}| から \verb|\end{document}| の間では、\verb|\include| コマンドを使うことで、別の {\tt *.tex} ファイルを読み込ませられる。 

\begin{itembox}[l]{{\tt include}しない場合}
\begin{itembox}[l]{{\tt main.tex}}
\begin{verbatim}
\begin{document}
  \begin{jabstract}
  ほげほげ
  \end{jabstract}
\end{document}
\end{verbatim}
\end{itembox}
\end{itembox}

\begin{itembox}[l]{{\tt include}する場合}
\begin{minipage}{0.5\hsize}
\begin{itembox}[l]{{\tt main.tex}}
\begin{verbatim}
\begin{document}
\chapter{序論}
\label{chap:introduction}

この章では、まずこの研究を選んだ背景を述べた後、本論文の構成を示す。

\section{背景}

近年、IT化が進みソフトウェアの社会での必要性がますます強くなり、開発規模も大きくなってきており、ソフトウェアはチームで開発されることが多い。また昨今OSSの種類も増え、既存のコードを再利用する機会は開発において少なくない。

\section{本文書の構成}

第\ref{chap:introduction}章では本研究の背景を書いた。第\ref{chap:related}章では、関連研究としてどのようなものがあるのかを示していく。第\ref{chap:design}章で研究の目的とどのようなものを実装するかを示す。第\ref{chap:implement}章では、具体的にどのように実装するかを示す。第\ref{chap:experiment}では評価方法と実際のアンケートの結果を分析する。第\ref{chap:conclusion}章で結論を示す。 % 01.texをinclude
\end{document}
\end{verbatim}
\end{itembox}
\end{minipage}
\begin{minipage}{0.5\hsize}
\begin{itembox}[l]{{\tt 01.tex}}
\begin{verbatim}
\begin{jabstract}
ほげほげ
\end{jabstract}
\end{verbatim}
\end{itembox}
\end{minipage}
\end{itembox}

{\tt include}しない場合とする場合を比較するとこのとおり。どちらも出力結果は一緒。{\tt include}する場合は、読み込ませたい箇所に、読み込ませたい{\tt *.tex}ファイルの名前を、拡張子を除いて \verb|\include| コマンドで書けばよい。

\verb|\include| コマンドを用いるか用いないかは、たぶん文書量や個人の好みに依る。例えば章ごとに別のファイルにしておけば、修正箇所を探すときの手間が多少は省けるかもしれない。Gitで人と共有しつつ校正を頼むときにもファイルが分かれていたほうがコンフリクトを起こしにくい。


\subsection{表紙の出力}

\begin{itembox}[l]{{\tt main.tex}}
\begin{verbatim}
\ifjapanese
  \jmaketitle    % 表紙(日本語)
\else
  \emaketitle    % 表紙(英語)
\fi
\end{verbatim}
\end{itembox}

最初に、表紙を出力する。

\verb|\jmaketitle| が実行されると日本語の表紙が、\verb|\emaketitle| が実行されると英語の表紙がそれぞれ出力される。日本語の表紙には、第\ref{sec:meta}節で設定したうちの日本語の情報が、英語の表紙には同節で設定したうち英語の情報が、それぞれ参照されて、表記される。

デフォルトでは第\ref{sec:lang}説で設定した言語の表紙のみが出力されるようになっている。

\subsection{アブストラクトの出力}

\begin{itembox}[l]{{\tt main.tex}}
\begin{verbatim}
% ■ アブストラクトの出力 ■
%	◆書式:
%		begin{jabstract}〜end{jabstract}	:日本語のアブストラクト
%		begin{eabstract}〜end{eabstract}	:英語のアブストラクト
%		※ 不要ならばコマンドごと消せば出力されない。



% 日本語のアブストラクト
\begin{jabstract}

ソフトウェア開発において、既存のソースコードを理解する能力はプロジェクトの結果にとても大きな影響を与える。また、昨今多くのOSSが開発されており、それらを利用する際にもソースコードを理解して利用することはとても大きなアドバンテージになる。ソースコードを理解できればドキュメントがまだ充実していないソフトウェアでも十分に活用できる。
しかし、ソースコードリーディングにはプログラムに対する深い理解や、処理を追う能力など多くの高度な能力を要求され、完全に理解するには時間と労力が必要になる。本論文ではpythonを例にとり、ソースコードを日本語に変換することによるソースコードリーディングへの影響をアンケートをもとに評価する。

\end{jabstract}



% 英語のアブストラクト
\begin{eabstract}

Eigo ga dekinai node Roma-ji de soreppoi hunniki wo daseruto iina.

Murippoi desu ne.

Write down your abstract here. Write down your abstract here. Write down your abstract here. Write down your abstract here. Write down your abstract here. Write down your abstract here.

 Write down your abstract here. Write down your abstract here. Write down your abstract here. Write down your abstract here. Write down your abstract here. Write down your abstract here. Write down your abstract here. Write down your abstract here. Write down your abstract here. Write down your abstract here. Write down your abstract here. Write down your abstract here.
 
Write down your abstract here. Write down your abstract here.

\end{eabstract}
	% アブストラクト。要独自コマンド、include先参照のこと
\end{verbatim}
\end{itembox}

表紙の次は、アブストラクト。

アブストラクトを出力するには、出力したい位置に、指定のコマンドを用いて文章を書き下せばよい。{\tt main.tex}に直接書いてもよいし、先述した \verb|\include| コマンドを利用して{\tt include}してもよい。

\verb|\begin{jabstract}| から \verb|\end{jabstract}| の間に書いた文章が日本語のアブストラクトとして、\verb|\begin{eabstract}| から \verb|\end{eabstract}| の間に書いた文章が英語のアブストラクトとして、それぞれ独立したページに出力される。

アブストラクトのページには、論文のタイトルやキーワードなどが、第\ref{sec:meta}節で設定した情報をもとにして自動で表記される。

日本語か英語のどちらか一方のみでよい場合は、不要な言語の方のコマンドを削除すればよい。これは、\verb|\begin| と \verb|\end| というコマンド自身も含めて削除する、ということで、\verb|\begin| と \verb|\end| の間を空っぽにするという意味ではないので注意。



\subsection{目次類の出力}
\label{sec:toc}

\begin{itembox}[l]{{\tt main.tex}}
\begin{verbatim}
\tableofcontents	% 目次
\listoffigures		% 表目次
\listoftables		% 図目次
\end{verbatim}
\end{itembox}

アブストラクトの次に、目次。文書の目次、図の目次、表の目次の三種類。

目次類を出力するには、出力したい位置に指定のコマンドを書けばよい。

これらのコマンドは、コンパイル時点での一時ファイル\footnote{{\tt *.toc}、{\tt *.lof}、{\tt *.lot}}の情報を、目次として体裁を整えて出力するもの。一時ファイルは、\verb|\begin{document}| から \verb|\end{document}| の間の章や節、図や表をコンパイルするときに、ついでに情報を取得しておいて生成される。

つまり気をつけなければいけないのは、コンパイルを一回しただけでは、一時ファイルが最新の状態に更新されるだけで、肝心の目次は正しい情報では出力されないということ。目次類を正しい情報で出力するには、最低二回のコンパイルが必要。一回目のコンパイルで一時ファイルが最新の情報に更新されて、二回目のコンパイルで初めて、その最新の一時ファイルの情報をもとに目次が出力される。

だから、文書に何らかの修正をして保存したあとは、最低でも二回、連続してコンパイルしないといけないことに注意する。

図や表を一つも使用していない場合は、目次名のみが書かれた空白のページが出力される。もしこれが不要な場合は、該当するコマンドをコメントアウトすればよい。


\subsection{本文の出力}

\begin{itembox}[l]{{\tt main.tex}}
\begin{verbatim}
\chapter{序論}
\label{chap:introduction}

この章では、まずこの研究を選んだ背景を述べた後、本論文の構成を示す。

\section{背景}

近年、IT化が進みソフトウェアの社会での必要性がますます強くなり、開発規模も大きくなってきており、ソフトウェアはチームで開発されることが多い。また昨今OSSの種類も増え、既存のコードを再利用する機会は開発において少なくない。

\section{本文書の構成}

第\ref{chap:introduction}章では本研究の背景を書いた。第\ref{chap:related}章では、関連研究としてどのようなものがあるのかを示していく。第\ref{chap:design}章で研究の目的とどのようなものを実装するかを示す。第\ref{chap:implement}章では、具体的にどのように実装するかを示す。第\ref{chap:experiment}では評価方法と実際のアンケートの結果を分析する。第\ref{chap:conclusion}章で結論を示す。	% 本文1
\chapter{関連研究}
\label{chap:related}


\section{テンプレートの構成}

このテンプレートは、表\ref{tb:files}のファイルで構成されている。

\begin{table}[htbp]
  \caption{構成ファイル}
  \label{tb:files}
  \begin{center}\begin{tabular}{c|l}
    \hline
    ファイル名&用途\\\hline\hline
    {\tt main.tex}&メインのファイル。これを編集していく\\\hline
    {\tt thesis.sty}&論文のスタイルを定義したファイル。基本的には手は加えない\\\hline
    {\tt *.tex}&{\tt main.tex}に{\tt include}されるファイル群\\\hline
    {\tt *.eps}&画像ファイル\\\hline
    {\tt main.bib}&参考文献用のBibTeXファイル\\\hline
    {\tt Makefile}&Makefile。次節以降で説明\\\hline
    {\tt .gitignore}&Git用設定ファイル\\\hline
  \end{tabular}\end{center}
\end{table}

\section{コンパイル}
このテンプレートの\LaTeX ファイルをコンパイルしてPDFファイルを生成するには、ターミナルを開いて以下のようにする。

\begin{itembox}[l]{コマンド実行例}
\begin{verbatim}
% make
\end{verbatim}
\end{itembox}

こうすることで、\verb|platex|コマンド、\verb|pbibtex|コマンド、\verb|platex|コマンド2回、\verb|dvipdfmx|コマンドが全て実行され、{\tt main.pdf}が生成される。

コンパイルによって生成されたファイルを全て消すには、以下のようにする。

\begin{itembox}[l]{コマンド実行例}
\begin{verbatim}
% make clean
\end{verbatim}
\end{itembox}

\section{設定}

以下、{\tt main.tex}に対して行うべき設定を、このファイルの中に書いてある順に沿って説明する。

\subsection{論文全体の言語の設定}
\label{sec:lang}

\begin{itembox}[l]{{\tt main.tex}}
\begin{verbatim}
\japanesetrue	% 論文全体を日本語で書く(英語で書くならコメントアウト)
\end{verbatim}
\end{itembox}

ここでは論文全体の言語を設定する。日本語に設定すれば、『章』『目次』『謝辞』などが日本語で出力されて、行頭のインデントなども日本語の仕様になる。英語にした場合は、これらはそれぞれ『Chapter』『Table of Contents』『Acknowledgment』な体裁になる。インデントも行間も、英語用の設定が適用される。

\verb|\japanesetrue| をコメントアウトしなければ日本語に、コメントアウトすれば英語に設定される。


\subsection{余白の設定}

\begin{itembox}[l]{{\tt main.tex}}
\begin{verbatim}
\bindermode	% バインダ用余白設定
\end{verbatim}
\end{itembox}

このテンプレートの出力はA4用紙。ここではこれの四辺の余白を設定する。

最終的にバインダーで綴じて提出する場合、余白を左右対称にしてしまうと、見かけ上のバランスがとても悪くなる。これを解消するため、あらかじめ左側の余白を大きく取っておく。

\verb|\bindermode| をコメントアウトしなければ左綴じ用の余白に、コメントアウトすれば左右対称の余白に設定される。

両面印刷の場合、偶数ページと奇数ページで余白を広くとるべき側が違うので、\verb|documentclass| でこれを設定する。

\begin{itembox}[l]{{\tt main.tex}}
\begin{verbatim}
% 両面印刷の場合。余白を綴じ側に作って右起こし。
\documentclass[a4j,twoside,openright,11pt]{jreport}
% 片面印刷の場合。
%\documentclass[a4j,11pt]{jreport}
\end{verbatim}
\end{itembox}

両面印刷の場合は \verb|twoside| を使用する。\verb|openright| を使うと章のはじまりが必ず右側のページに来るようになる。

\subsection{論文情報の設定}
\label{sec:meta}

\begin{itembox}[l]{{\tt main.tex}}
\begin{verbatim}
% 日本語情報(必要なら)
\jclass  {修士論文}                             % 論文種別
\jtitle    {修士論文用 \LaTeX\ テンプレート}    % タイトル。改行する場合は\\を入れる
\juniv    {慶應義塾大学大学院}                  % 大学名
\jfaculty  {政策・メディア研究科}               % 学部、学科
\jauthor  {ほげ山 ふう助}                       % 著者
\jhyear  {24}                                   % 平成○年度
\jsyear  {2012}                                 % 西暦○年度
\jkeyword  {\LaTeX、テンプレート、修士論文}     % 論文のキーワード
\jproject{インタラクションデザインプロジェクト} %プロジェクト名
\jdate{2013年1月}

% 英語情報(必要なら)
\eclass  {Master's Thesis}                            % 論文種別
\etitle    {A \LaTeX Template for Master Thesis}      % タイトル。改行する場合は\\を入れる
\euniv  {Keio University}                             % 大学名
\efaculty  {Graduate School of Media and Governance}  % 学部、学科
\eauthor  {Fusuke Hogeyama}                           % 著者
\eyear  {2012}                                        % 西暦○年度
\ekeyword  {\LaTeX, Templete, Master Thesis}          % 論文のキーワード
\eproject{Interaction Design Project}                 %プロジェクト名
\edate{January 2013}
\end{verbatim}
\end{itembox}

ここでは論文のタイトルや著者の氏名などのメタデータを記述する。ここで書いたデータは、表紙とアブストラクトのページに使われる。必ずしも日本語と英語の両方を設定しなければいけないわけではなくて、自分が必要とする方だけ記述すればよい。

タイトルが長過ぎる場合は、表紙やアブストラクトのページでは自動で折り返して出力される。もし改行位置を自分で指定したい場合は、その場所に \verb|\\| を入力する。


\section{出力}

\verb|\begin{document}| から \verb|\end{document}| に記述した部分が、実際に{\tt DVI}(最終的には{\tt PDF})ファイルとして出力される。

\subsection{外部ファイルの読み込み({\tt include})}

出力部分の具体的な説明の前に、外部ファイルを読み込む方法を説明する。

\verb|\begin{document}| から \verb|\end{document}| の間では、\verb|\include| コマンドを使うことで、別の {\tt *.tex} ファイルを読み込ませられる。 

\begin{itembox}[l]{{\tt include}しない場合}
\begin{itembox}[l]{{\tt main.tex}}
\begin{verbatim}
\begin{document}
  \begin{jabstract}
  ほげほげ
  \end{jabstract}
\end{document}
\end{verbatim}
\end{itembox}
\end{itembox}

\begin{itembox}[l]{{\tt include}する場合}
\begin{minipage}{0.5\hsize}
\begin{itembox}[l]{{\tt main.tex}}
\begin{verbatim}
\begin{document}
\chapter{序論}
\label{chap:introduction}

この章では、まずこの研究を選んだ背景を述べた後、本論文の構成を示す。

\section{背景}

近年、IT化が進みソフトウェアの社会での必要性がますます強くなり、開発規模も大きくなってきており、ソフトウェアはチームで開発されることが多い。また昨今OSSの種類も増え、既存のコードを再利用する機会は開発において少なくない。

\section{本文書の構成}

第\ref{chap:introduction}章では本研究の背景を書いた。第\ref{chap:related}章では、関連研究としてどのようなものがあるのかを示していく。第\ref{chap:design}章で研究の目的とどのようなものを実装するかを示す。第\ref{chap:implement}章では、具体的にどのように実装するかを示す。第\ref{chap:experiment}では評価方法と実際のアンケートの結果を分析する。第\ref{chap:conclusion}章で結論を示す。 % 01.texをinclude
\end{document}
\end{verbatim}
\end{itembox}
\end{minipage}
\begin{minipage}{0.5\hsize}
\begin{itembox}[l]{{\tt 01.tex}}
\begin{verbatim}
\begin{jabstract}
ほげほげ
\end{jabstract}
\end{verbatim}
\end{itembox}
\end{minipage}
\end{itembox}

{\tt include}しない場合とする場合を比較するとこのとおり。どちらも出力結果は一緒。{\tt include}する場合は、読み込ませたい箇所に、読み込ませたい{\tt *.tex}ファイルの名前を、拡張子を除いて \verb|\include| コマンドで書けばよい。

\verb|\include| コマンドを用いるか用いないかは、たぶん文書量や個人の好みに依る。例えば章ごとに別のファイルにしておけば、修正箇所を探すときの手間が多少は省けるかもしれない。Gitで人と共有しつつ校正を頼むときにもファイルが分かれていたほうがコンフリクトを起こしにくい。


\subsection{表紙の出力}

\begin{itembox}[l]{{\tt main.tex}}
\begin{verbatim}
\ifjapanese
  \jmaketitle    % 表紙(日本語)
\else
  \emaketitle    % 表紙(英語)
\fi
\end{verbatim}
\end{itembox}

最初に、表紙を出力する。

\verb|\jmaketitle| が実行されると日本語の表紙が、\verb|\emaketitle| が実行されると英語の表紙がそれぞれ出力される。日本語の表紙には、第\ref{sec:meta}節で設定したうちの日本語の情報が、英語の表紙には同節で設定したうち英語の情報が、それぞれ参照されて、表記される。

デフォルトでは第\ref{sec:lang}説で設定した言語の表紙のみが出力されるようになっている。

\subsection{アブストラクトの出力}

\begin{itembox}[l]{{\tt main.tex}}
\begin{verbatim}
% ■ アブストラクトの出力 ■
%	◆書式:
%		begin{jabstract}〜end{jabstract}	:日本語のアブストラクト
%		begin{eabstract}〜end{eabstract}	:英語のアブストラクト
%		※ 不要ならばコマンドごと消せば出力されない。



% 日本語のアブストラクト
\begin{jabstract}

ソフトウェア開発において、既存のソースコードを理解する能力はプロジェクトの結果にとても大きな影響を与える。また、昨今多くのOSSが開発されており、それらを利用する際にもソースコードを理解して利用することはとても大きなアドバンテージになる。ソースコードを理解できればドキュメントがまだ充実していないソフトウェアでも十分に活用できる。
しかし、ソースコードリーディングにはプログラムに対する深い理解や、処理を追う能力など多くの高度な能力を要求され、完全に理解するには時間と労力が必要になる。本論文ではpythonを例にとり、ソースコードを日本語に変換することによるソースコードリーディングへの影響をアンケートをもとに評価する。

\end{jabstract}



% 英語のアブストラクト
\begin{eabstract}

Eigo ga dekinai node Roma-ji de soreppoi hunniki wo daseruto iina.

Murippoi desu ne.

Write down your abstract here. Write down your abstract here. Write down your abstract here. Write down your abstract here. Write down your abstract here. Write down your abstract here.

 Write down your abstract here. Write down your abstract here. Write down your abstract here. Write down your abstract here. Write down your abstract here. Write down your abstract here. Write down your abstract here. Write down your abstract here. Write down your abstract here. Write down your abstract here. Write down your abstract here. Write down your abstract here.
 
Write down your abstract here. Write down your abstract here.

\end{eabstract}
	% アブストラクト。要独自コマンド、include先参照のこと
\end{verbatim}
\end{itembox}

表紙の次は、アブストラクト。

アブストラクトを出力するには、出力したい位置に、指定のコマンドを用いて文章を書き下せばよい。{\tt main.tex}に直接書いてもよいし、先述した \verb|\include| コマンドを利用して{\tt include}してもよい。

\verb|\begin{jabstract}| から \verb|\end{jabstract}| の間に書いた文章が日本語のアブストラクトとして、\verb|\begin{eabstract}| から \verb|\end{eabstract}| の間に書いた文章が英語のアブストラクトとして、それぞれ独立したページに出力される。

アブストラクトのページには、論文のタイトルやキーワードなどが、第\ref{sec:meta}節で設定した情報をもとにして自動で表記される。

日本語か英語のどちらか一方のみでよい場合は、不要な言語の方のコマンドを削除すればよい。これは、\verb|\begin| と \verb|\end| というコマンド自身も含めて削除する、ということで、\verb|\begin| と \verb|\end| の間を空っぽにするという意味ではないので注意。



\subsection{目次類の出力}
\label{sec:toc}

\begin{itembox}[l]{{\tt main.tex}}
\begin{verbatim}
\tableofcontents	% 目次
\listoffigures		% 表目次
\listoftables		% 図目次
\end{verbatim}
\end{itembox}

アブストラクトの次に、目次。文書の目次、図の目次、表の目次の三種類。

目次類を出力するには、出力したい位置に指定のコマンドを書けばよい。

これらのコマンドは、コンパイル時点での一時ファイル\footnote{{\tt *.toc}、{\tt *.lof}、{\tt *.lot}}の情報を、目次として体裁を整えて出力するもの。一時ファイルは、\verb|\begin{document}| から \verb|\end{document}| の間の章や節、図や表をコンパイルするときに、ついでに情報を取得しておいて生成される。

つまり気をつけなければいけないのは、コンパイルを一回しただけでは、一時ファイルが最新の状態に更新されるだけで、肝心の目次は正しい情報では出力されないということ。目次類を正しい情報で出力するには、最低二回のコンパイルが必要。一回目のコンパイルで一時ファイルが最新の情報に更新されて、二回目のコンパイルで初めて、その最新の一時ファイルの情報をもとに目次が出力される。

だから、文書に何らかの修正をして保存したあとは、最低でも二回、連続してコンパイルしないといけないことに注意する。

図や表を一つも使用していない場合は、目次名のみが書かれた空白のページが出力される。もしこれが不要な場合は、該当するコマンドをコメントアウトすればよい。


\subsection{本文の出力}

\begin{itembox}[l]{{\tt main.tex}}
\begin{verbatim}
\chapter{序論}
\label{chap:introduction}

この章では、まずこの研究を選んだ背景を述べた後、本論文の構成を示す。

\section{背景}

近年、IT化が進みソフトウェアの社会での必要性がますます強くなり、開発規模も大きくなってきており、ソフトウェアはチームで開発されることが多い。また昨今OSSの種類も増え、既存のコードを再利用する機会は開発において少なくない。

\section{本文書の構成}

第\ref{chap:introduction}章では本研究の背景を書いた。第\ref{chap:related}章では、関連研究としてどのようなものがあるのかを示していく。第\ref{chap:design}章で研究の目的とどのようなものを実装するかを示す。第\ref{chap:implement}章では、具体的にどのように実装するかを示す。第\ref{chap:experiment}では評価方法と実際のアンケートの結果を分析する。第\ref{chap:conclusion}章で結論を示す。	% 本文1
\chapter{関連研究}
\label{chap:related}


\section{テンプレートの構成}

このテンプレートは、表\ref{tb:files}のファイルで構成されている。

\begin{table}[htbp]
  \caption{構成ファイル}
  \label{tb:files}
  \begin{center}\begin{tabular}{c|l}
    \hline
    ファイル名&用途\\\hline\hline
    {\tt main.tex}&メインのファイル。これを編集していく\\\hline
    {\tt thesis.sty}&論文のスタイルを定義したファイル。基本的には手は加えない\\\hline
    {\tt *.tex}&{\tt main.tex}に{\tt include}されるファイル群\\\hline
    {\tt *.eps}&画像ファイル\\\hline
    {\tt main.bib}&参考文献用のBibTeXファイル\\\hline
    {\tt Makefile}&Makefile。次節以降で説明\\\hline
    {\tt .gitignore}&Git用設定ファイル\\\hline
  \end{tabular}\end{center}
\end{table}

\section{コンパイル}
このテンプレートの\LaTeX ファイルをコンパイルしてPDFファイルを生成するには、ターミナルを開いて以下のようにする。

\begin{itembox}[l]{コマンド実行例}
\begin{verbatim}
% make
\end{verbatim}
\end{itembox}

こうすることで、\verb|platex|コマンド、\verb|pbibtex|コマンド、\verb|platex|コマンド2回、\verb|dvipdfmx|コマンドが全て実行され、{\tt main.pdf}が生成される。

コンパイルによって生成されたファイルを全て消すには、以下のようにする。

\begin{itembox}[l]{コマンド実行例}
\begin{verbatim}
% make clean
\end{verbatim}
\end{itembox}

\section{設定}

以下、{\tt main.tex}に対して行うべき設定を、このファイルの中に書いてある順に沿って説明する。

\subsection{論文全体の言語の設定}
\label{sec:lang}

\begin{itembox}[l]{{\tt main.tex}}
\begin{verbatim}
\japanesetrue	% 論文全体を日本語で書く(英語で書くならコメントアウト)
\end{verbatim}
\end{itembox}

ここでは論文全体の言語を設定する。日本語に設定すれば、『章』『目次』『謝辞』などが日本語で出力されて、行頭のインデントなども日本語の仕様になる。英語にした場合は、これらはそれぞれ『Chapter』『Table of Contents』『Acknowledgment』な体裁になる。インデントも行間も、英語用の設定が適用される。

\verb|\japanesetrue| をコメントアウトしなければ日本語に、コメントアウトすれば英語に設定される。


\subsection{余白の設定}

\begin{itembox}[l]{{\tt main.tex}}
\begin{verbatim}
\bindermode	% バインダ用余白設定
\end{verbatim}
\end{itembox}

このテンプレートの出力はA4用紙。ここではこれの四辺の余白を設定する。

最終的にバインダーで綴じて提出する場合、余白を左右対称にしてしまうと、見かけ上のバランスがとても悪くなる。これを解消するため、あらかじめ左側の余白を大きく取っておく。

\verb|\bindermode| をコメントアウトしなければ左綴じ用の余白に、コメントアウトすれば左右対称の余白に設定される。

両面印刷の場合、偶数ページと奇数ページで余白を広くとるべき側が違うので、\verb|documentclass| でこれを設定する。

\begin{itembox}[l]{{\tt main.tex}}
\begin{verbatim}
% 両面印刷の場合。余白を綴じ側に作って右起こし。
\documentclass[a4j,twoside,openright,11pt]{jreport}
% 片面印刷の場合。
%\documentclass[a4j,11pt]{jreport}
\end{verbatim}
\end{itembox}

両面印刷の場合は \verb|twoside| を使用する。\verb|openright| を使うと章のはじまりが必ず右側のページに来るようになる。

\subsection{論文情報の設定}
\label{sec:meta}

\begin{itembox}[l]{{\tt main.tex}}
\begin{verbatim}
% 日本語情報(必要なら)
\jclass  {修士論文}                             % 論文種別
\jtitle    {修士論文用 \LaTeX\ テンプレート}    % タイトル。改行する場合は\\を入れる
\juniv    {慶應義塾大学大学院}                  % 大学名
\jfaculty  {政策・メディア研究科}               % 学部、学科
\jauthor  {ほげ山 ふう助}                       % 著者
\jhyear  {24}                                   % 平成○年度
\jsyear  {2012}                                 % 西暦○年度
\jkeyword  {\LaTeX、テンプレート、修士論文}     % 論文のキーワード
\jproject{インタラクションデザインプロジェクト} %プロジェクト名
\jdate{2013年1月}

% 英語情報(必要なら)
\eclass  {Master's Thesis}                            % 論文種別
\etitle    {A \LaTeX Template for Master Thesis}      % タイトル。改行する場合は\\を入れる
\euniv  {Keio University}                             % 大学名
\efaculty  {Graduate School of Media and Governance}  % 学部、学科
\eauthor  {Fusuke Hogeyama}                           % 著者
\eyear  {2012}                                        % 西暦○年度
\ekeyword  {\LaTeX, Templete, Master Thesis}          % 論文のキーワード
\eproject{Interaction Design Project}                 %プロジェクト名
\edate{January 2013}
\end{verbatim}
\end{itembox}

ここでは論文のタイトルや著者の氏名などのメタデータを記述する。ここで書いたデータは、表紙とアブストラクトのページに使われる。必ずしも日本語と英語の両方を設定しなければいけないわけではなくて、自分が必要とする方だけ記述すればよい。

タイトルが長過ぎる場合は、表紙やアブストラクトのページでは自動で折り返して出力される。もし改行位置を自分で指定したい場合は、その場所に \verb|\\| を入力する。


\section{出力}

\verb|\begin{document}| から \verb|\end{document}| に記述した部分が、実際に{\tt DVI}(最終的には{\tt PDF})ファイルとして出力される。

\subsection{外部ファイルの読み込み({\tt include})}

出力部分の具体的な説明の前に、外部ファイルを読み込む方法を説明する。

\verb|\begin{document}| から \verb|\end{document}| の間では、\verb|\include| コマンドを使うことで、別の {\tt *.tex} ファイルを読み込ませられる。 

\begin{itembox}[l]{{\tt include}しない場合}
\begin{itembox}[l]{{\tt main.tex}}
\begin{verbatim}
\begin{document}
  \begin{jabstract}
  ほげほげ
  \end{jabstract}
\end{document}
\end{verbatim}
\end{itembox}
\end{itembox}

\begin{itembox}[l]{{\tt include}する場合}
\begin{minipage}{0.5\hsize}
\begin{itembox}[l]{{\tt main.tex}}
\begin{verbatim}
\begin{document}
\chapter{序論}
\label{chap:introduction}

この章では、まずこの研究を選んだ背景を述べた後、本論文の構成を示す。

\section{背景}

近年、IT化が進みソフトウェアの社会での必要性がますます強くなり、開発規模も大きくなってきており、ソフトウェアはチームで開発されることが多い。また昨今OSSの種類も増え、既存のコードを再利用する機会は開発において少なくない。

\section{本文書の構成}

第\ref{chap:introduction}章では本研究の背景を書いた。第\ref{chap:related}章では、関連研究としてどのようなものがあるのかを示していく。第\ref{chap:design}章で研究の目的とどのようなものを実装するかを示す。第\ref{chap:implement}章では、具体的にどのように実装するかを示す。第\ref{chap:experiment}では評価方法と実際のアンケートの結果を分析する。第\ref{chap:conclusion}章で結論を示す。 % 01.texをinclude
\end{document}
\end{verbatim}
\end{itembox}
\end{minipage}
\begin{minipage}{0.5\hsize}
\begin{itembox}[l]{{\tt 01.tex}}
\begin{verbatim}
\begin{jabstract}
ほげほげ
\end{jabstract}
\end{verbatim}
\end{itembox}
\end{minipage}
\end{itembox}

{\tt include}しない場合とする場合を比較するとこのとおり。どちらも出力結果は一緒。{\tt include}する場合は、読み込ませたい箇所に、読み込ませたい{\tt *.tex}ファイルの名前を、拡張子を除いて \verb|\include| コマンドで書けばよい。

\verb|\include| コマンドを用いるか用いないかは、たぶん文書量や個人の好みに依る。例えば章ごとに別のファイルにしておけば、修正箇所を探すときの手間が多少は省けるかもしれない。Gitで人と共有しつつ校正を頼むときにもファイルが分かれていたほうがコンフリクトを起こしにくい。


\subsection{表紙の出力}

\begin{itembox}[l]{{\tt main.tex}}
\begin{verbatim}
\ifjapanese
  \jmaketitle    % 表紙(日本語)
\else
  \emaketitle    % 表紙(英語)
\fi
\end{verbatim}
\end{itembox}

最初に、表紙を出力する。

\verb|\jmaketitle| が実行されると日本語の表紙が、\verb|\emaketitle| が実行されると英語の表紙がそれぞれ出力される。日本語の表紙には、第\ref{sec:meta}節で設定したうちの日本語の情報が、英語の表紙には同節で設定したうち英語の情報が、それぞれ参照されて、表記される。

デフォルトでは第\ref{sec:lang}説で設定した言語の表紙のみが出力されるようになっている。

\subsection{アブストラクトの出力}

\begin{itembox}[l]{{\tt main.tex}}
\begin{verbatim}
% ■ アブストラクトの出力 ■
%	◆書式:
%		begin{jabstract}〜end{jabstract}	:日本語のアブストラクト
%		begin{eabstract}〜end{eabstract}	:英語のアブストラクト
%		※ 不要ならばコマンドごと消せば出力されない。



% 日本語のアブストラクト
\begin{jabstract}

ソフトウェア開発において、既存のソースコードを理解する能力はプロジェクトの結果にとても大きな影響を与える。また、昨今多くのOSSが開発されており、それらを利用する際にもソースコードを理解して利用することはとても大きなアドバンテージになる。ソースコードを理解できればドキュメントがまだ充実していないソフトウェアでも十分に活用できる。
しかし、ソースコードリーディングにはプログラムに対する深い理解や、処理を追う能力など多くの高度な能力を要求され、完全に理解するには時間と労力が必要になる。本論文ではpythonを例にとり、ソースコードを日本語に変換することによるソースコードリーディングへの影響をアンケートをもとに評価する。

\end{jabstract}



% 英語のアブストラクト
\begin{eabstract}

Eigo ga dekinai node Roma-ji de soreppoi hunniki wo daseruto iina.

Murippoi desu ne.

Write down your abstract here. Write down your abstract here. Write down your abstract here. Write down your abstract here. Write down your abstract here. Write down your abstract here.

 Write down your abstract here. Write down your abstract here. Write down your abstract here. Write down your abstract here. Write down your abstract here. Write down your abstract here. Write down your abstract here. Write down your abstract here. Write down your abstract here. Write down your abstract here. Write down your abstract here. Write down your abstract here.
 
Write down your abstract here. Write down your abstract here.

\end{eabstract}
	% アブストラクト。要独自コマンド、include先参照のこと
\end{verbatim}
\end{itembox}

表紙の次は、アブストラクト。

アブストラクトを出力するには、出力したい位置に、指定のコマンドを用いて文章を書き下せばよい。{\tt main.tex}に直接書いてもよいし、先述した \verb|\include| コマンドを利用して{\tt include}してもよい。

\verb|\begin{jabstract}| から \verb|\end{jabstract}| の間に書いた文章が日本語のアブストラクトとして、\verb|\begin{eabstract}| から \verb|\end{eabstract}| の間に書いた文章が英語のアブストラクトとして、それぞれ独立したページに出力される。

アブストラクトのページには、論文のタイトルやキーワードなどが、第\ref{sec:meta}節で設定した情報をもとにして自動で表記される。

日本語か英語のどちらか一方のみでよい場合は、不要な言語の方のコマンドを削除すればよい。これは、\verb|\begin| と \verb|\end| というコマンド自身も含めて削除する、ということで、\verb|\begin| と \verb|\end| の間を空っぽにするという意味ではないので注意。



\subsection{目次類の出力}
\label{sec:toc}

\begin{itembox}[l]{{\tt main.tex}}
\begin{verbatim}
\tableofcontents	% 目次
\listoffigures		% 表目次
\listoftables		% 図目次
\end{verbatim}
\end{itembox}

アブストラクトの次に、目次。文書の目次、図の目次、表の目次の三種類。

目次類を出力するには、出力したい位置に指定のコマンドを書けばよい。

これらのコマンドは、コンパイル時点での一時ファイル\footnote{{\tt *.toc}、{\tt *.lof}、{\tt *.lot}}の情報を、目次として体裁を整えて出力するもの。一時ファイルは、\verb|\begin{document}| から \verb|\end{document}| の間の章や節、図や表をコンパイルするときに、ついでに情報を取得しておいて生成される。

つまり気をつけなければいけないのは、コンパイルを一回しただけでは、一時ファイルが最新の状態に更新されるだけで、肝心の目次は正しい情報では出力されないということ。目次類を正しい情報で出力するには、最低二回のコンパイルが必要。一回目のコンパイルで一時ファイルが最新の情報に更新されて、二回目のコンパイルで初めて、その最新の一時ファイルの情報をもとに目次が出力される。

だから、文書に何らかの修正をして保存したあとは、最低でも二回、連続してコンパイルしないといけないことに注意する。

図や表を一つも使用していない場合は、目次名のみが書かれた空白のページが出力される。もしこれが不要な場合は、該当するコマンドをコメントアウトすればよい。


\subsection{本文の出力}

\begin{itembox}[l]{{\tt main.tex}}
\begin{verbatim}
\chapter{序論}
\label{chap:introduction}

この章では、まずこの研究を選んだ背景を述べた後、本論文の構成を示す。

\section{背景}

近年、IT化が進みソフトウェアの社会での必要性がますます強くなり、開発規模も大きくなってきており、ソフトウェアはチームで開発されることが多い。また昨今OSSの種類も増え、既存のコードを再利用する機会は開発において少なくない。

\section{本文書の構成}

第\ref{chap:introduction}章では本研究の背景を書いた。第\ref{chap:related}章では、関連研究としてどのようなものがあるのかを示していく。第\ref{chap:design}章で研究の目的とどのようなものを実装するかを示す。第\ref{chap:implement}章では、具体的にどのように実装するかを示す。第\ref{chap:experiment}では評価方法と実際のアンケートの結果を分析する。第\ref{chap:conclusion}章で結論を示す。	% 本文1
\chapter{関連研究}
\label{chap:related}


\section{テンプレートの構成}

このテンプレートは、表\ref{tb:files}のファイルで構成されている。

\begin{table}[htbp]
  \caption{構成ファイル}
  \label{tb:files}
  \begin{center}\begin{tabular}{c|l}
    \hline
    ファイル名&用途\\\hline\hline
    {\tt main.tex}&メインのファイル。これを編集していく\\\hline
    {\tt thesis.sty}&論文のスタイルを定義したファイル。基本的には手は加えない\\\hline
    {\tt *.tex}&{\tt main.tex}に{\tt include}されるファイル群\\\hline
    {\tt *.eps}&画像ファイル\\\hline
    {\tt main.bib}&参考文献用のBibTeXファイル\\\hline
    {\tt Makefile}&Makefile。次節以降で説明\\\hline
    {\tt .gitignore}&Git用設定ファイル\\\hline
  \end{tabular}\end{center}
\end{table}

\section{コンパイル}
このテンプレートの\LaTeX ファイルをコンパイルしてPDFファイルを生成するには、ターミナルを開いて以下のようにする。

\begin{itembox}[l]{コマンド実行例}
\begin{verbatim}
% make
\end{verbatim}
\end{itembox}

こうすることで、\verb|platex|コマンド、\verb|pbibtex|コマンド、\verb|platex|コマンド2回、\verb|dvipdfmx|コマンドが全て実行され、{\tt main.pdf}が生成される。

コンパイルによって生成されたファイルを全て消すには、以下のようにする。

\begin{itembox}[l]{コマンド実行例}
\begin{verbatim}
% make clean
\end{verbatim}
\end{itembox}

\section{設定}

以下、{\tt main.tex}に対して行うべき設定を、このファイルの中に書いてある順に沿って説明する。

\subsection{論文全体の言語の設定}
\label{sec:lang}

\begin{itembox}[l]{{\tt main.tex}}
\begin{verbatim}
\japanesetrue	% 論文全体を日本語で書く(英語で書くならコメントアウト)
\end{verbatim}
\end{itembox}

ここでは論文全体の言語を設定する。日本語に設定すれば、『章』『目次』『謝辞』などが日本語で出力されて、行頭のインデントなども日本語の仕様になる。英語にした場合は、これらはそれぞれ『Chapter』『Table of Contents』『Acknowledgment』な体裁になる。インデントも行間も、英語用の設定が適用される。

\verb|\japanesetrue| をコメントアウトしなければ日本語に、コメントアウトすれば英語に設定される。


\subsection{余白の設定}

\begin{itembox}[l]{{\tt main.tex}}
\begin{verbatim}
\bindermode	% バインダ用余白設定
\end{verbatim}
\end{itembox}

このテンプレートの出力はA4用紙。ここではこれの四辺の余白を設定する。

最終的にバインダーで綴じて提出する場合、余白を左右対称にしてしまうと、見かけ上のバランスがとても悪くなる。これを解消するため、あらかじめ左側の余白を大きく取っておく。

\verb|\bindermode| をコメントアウトしなければ左綴じ用の余白に、コメントアウトすれば左右対称の余白に設定される。

両面印刷の場合、偶数ページと奇数ページで余白を広くとるべき側が違うので、\verb|documentclass| でこれを設定する。

\begin{itembox}[l]{{\tt main.tex}}
\begin{verbatim}
% 両面印刷の場合。余白を綴じ側に作って右起こし。
\documentclass[a4j,twoside,openright,11pt]{jreport}
% 片面印刷の場合。
%\documentclass[a4j,11pt]{jreport}
\end{verbatim}
\end{itembox}

両面印刷の場合は \verb|twoside| を使用する。\verb|openright| を使うと章のはじまりが必ず右側のページに来るようになる。

\subsection{論文情報の設定}
\label{sec:meta}

\begin{itembox}[l]{{\tt main.tex}}
\begin{verbatim}
% 日本語情報(必要なら)
\jclass  {修士論文}                             % 論文種別
\jtitle    {修士論文用 \LaTeX\ テンプレート}    % タイトル。改行する場合は\\を入れる
\juniv    {慶應義塾大学大学院}                  % 大学名
\jfaculty  {政策・メディア研究科}               % 学部、学科
\jauthor  {ほげ山 ふう助}                       % 著者
\jhyear  {24}                                   % 平成○年度
\jsyear  {2012}                                 % 西暦○年度
\jkeyword  {\LaTeX、テンプレート、修士論文}     % 論文のキーワード
\jproject{インタラクションデザインプロジェクト} %プロジェクト名
\jdate{2013年1月}

% 英語情報(必要なら)
\eclass  {Master's Thesis}                            % 論文種別
\etitle    {A \LaTeX Template for Master Thesis}      % タイトル。改行する場合は\\を入れる
\euniv  {Keio University}                             % 大学名
\efaculty  {Graduate School of Media and Governance}  % 学部、学科
\eauthor  {Fusuke Hogeyama}                           % 著者
\eyear  {2012}                                        % 西暦○年度
\ekeyword  {\LaTeX, Templete, Master Thesis}          % 論文のキーワード
\eproject{Interaction Design Project}                 %プロジェクト名
\edate{January 2013}
\end{verbatim}
\end{itembox}

ここでは論文のタイトルや著者の氏名などのメタデータを記述する。ここで書いたデータは、表紙とアブストラクトのページに使われる。必ずしも日本語と英語の両方を設定しなければいけないわけではなくて、自分が必要とする方だけ記述すればよい。

タイトルが長過ぎる場合は、表紙やアブストラクトのページでは自動で折り返して出力される。もし改行位置を自分で指定したい場合は、その場所に \verb|\\| を入力する。


\section{出力}

\verb|\begin{document}| から \verb|\end{document}| に記述した部分が、実際に{\tt DVI}(最終的には{\tt PDF})ファイルとして出力される。

\subsection{外部ファイルの読み込み({\tt include})}

出力部分の具体的な説明の前に、外部ファイルを読み込む方法を説明する。

\verb|\begin{document}| から \verb|\end{document}| の間では、\verb|\include| コマンドを使うことで、別の {\tt *.tex} ファイルを読み込ませられる。 

\begin{itembox}[l]{{\tt include}しない場合}
\begin{itembox}[l]{{\tt main.tex}}
\begin{verbatim}
\begin{document}
  \begin{jabstract}
  ほげほげ
  \end{jabstract}
\end{document}
\end{verbatim}
\end{itembox}
\end{itembox}

\begin{itembox}[l]{{\tt include}する場合}
\begin{minipage}{0.5\hsize}
\begin{itembox}[l]{{\tt main.tex}}
\begin{verbatim}
\begin{document}
\include{01} % 01.texをinclude
\end{document}
\end{verbatim}
\end{itembox}
\end{minipage}
\begin{minipage}{0.5\hsize}
\begin{itembox}[l]{{\tt 01.tex}}
\begin{verbatim}
\begin{jabstract}
ほげほげ
\end{jabstract}
\end{verbatim}
\end{itembox}
\end{minipage}
\end{itembox}

{\tt include}しない場合とする場合を比較するとこのとおり。どちらも出力結果は一緒。{\tt include}する場合は、読み込ませたい箇所に、読み込ませたい{\tt *.tex}ファイルの名前を、拡張子を除いて \verb|\include| コマンドで書けばよい。

\verb|\include| コマンドを用いるか用いないかは、たぶん文書量や個人の好みに依る。例えば章ごとに別のファイルにしておけば、修正箇所を探すときの手間が多少は省けるかもしれない。Gitで人と共有しつつ校正を頼むときにもファイルが分かれていたほうがコンフリクトを起こしにくい。


\subsection{表紙の出力}

\begin{itembox}[l]{{\tt main.tex}}
\begin{verbatim}
\ifjapanese
  \jmaketitle    % 表紙(日本語)
\else
  \emaketitle    % 表紙(英語)
\fi
\end{verbatim}
\end{itembox}

最初に、表紙を出力する。

\verb|\jmaketitle| が実行されると日本語の表紙が、\verb|\emaketitle| が実行されると英語の表紙がそれぞれ出力される。日本語の表紙には、第\ref{sec:meta}節で設定したうちの日本語の情報が、英語の表紙には同節で設定したうち英語の情報が、それぞれ参照されて、表記される。

デフォルトでは第\ref{sec:lang}説で設定した言語の表紙のみが出力されるようになっている。

\subsection{アブストラクトの出力}

\begin{itembox}[l]{{\tt main.tex}}
\begin{verbatim}
\include{00_abstract}	% アブストラクト。要独自コマンド、include先参照のこと
\end{verbatim}
\end{itembox}

表紙の次は、アブストラクト。

アブストラクトを出力するには、出力したい位置に、指定のコマンドを用いて文章を書き下せばよい。{\tt main.tex}に直接書いてもよいし、先述した \verb|\include| コマンドを利用して{\tt include}してもよい。

\verb|\begin{jabstract}| から \verb|\end{jabstract}| の間に書いた文章が日本語のアブストラクトとして、\verb|\begin{eabstract}| から \verb|\end{eabstract}| の間に書いた文章が英語のアブストラクトとして、それぞれ独立したページに出力される。

アブストラクトのページには、論文のタイトルやキーワードなどが、第\ref{sec:meta}節で設定した情報をもとにして自動で表記される。

日本語か英語のどちらか一方のみでよい場合は、不要な言語の方のコマンドを削除すればよい。これは、\verb|\begin| と \verb|\end| というコマンド自身も含めて削除する、ということで、\verb|\begin| と \verb|\end| の間を空っぽにするという意味ではないので注意。



\subsection{目次類の出力}
\label{sec:toc}

\begin{itembox}[l]{{\tt main.tex}}
\begin{verbatim}
\tableofcontents	% 目次
\listoffigures		% 表目次
\listoftables		% 図目次
\end{verbatim}
\end{itembox}

アブストラクトの次に、目次。文書の目次、図の目次、表の目次の三種類。

目次類を出力するには、出力したい位置に指定のコマンドを書けばよい。

これらのコマンドは、コンパイル時点での一時ファイル\footnote{{\tt *.toc}、{\tt *.lof}、{\tt *.lot}}の情報を、目次として体裁を整えて出力するもの。一時ファイルは、\verb|\begin{document}| から \verb|\end{document}| の間の章や節、図や表をコンパイルするときに、ついでに情報を取得しておいて生成される。

つまり気をつけなければいけないのは、コンパイルを一回しただけでは、一時ファイルが最新の状態に更新されるだけで、肝心の目次は正しい情報では出力されないということ。目次類を正しい情報で出力するには、最低二回のコンパイルが必要。一回目のコンパイルで一時ファイルが最新の情報に更新されて、二回目のコンパイルで初めて、その最新の一時ファイルの情報をもとに目次が出力される。

だから、文書に何らかの修正をして保存したあとは、最低でも二回、連続してコンパイルしないといけないことに注意する。

図や表を一つも使用していない場合は、目次名のみが書かれた空白のページが出力される。もしこれが不要な場合は、該当するコマンドをコメントアウトすればよい。


\subsection{本文の出力}

\begin{itembox}[l]{{\tt main.tex}}
\begin{verbatim}
\include{01}	% 本文1
\include{02}	% 本文2
\include{03}	% 本文3
\include{04}	% 本文4
\end{verbatim}
\end{itembox}

目次に続いて、論文のメイン、本文を記述する。アブストラクトと同様で、{\tt main.tex}に直接書くか、\verb|\include| コマンドを利用して別に用意したファイルを{\tt include}する。

本文の書き方は、第\ref{chap:latex}章で詳しく説明する。


\subsection{謝辞の出力}

\begin{itembox}[l]{{\tt main.tex}}
\begin{verbatim}
\include{90_acknowledgment}	% 謝辞。要独自コマンド、include先参照のこと
\end{verbatim}
\end{itembox}

本文のあとには、謝辞を出力する。\verb|begin{acknowledgment}| から \verb|end{acknowledgment}| の間に書いた文章が、謝辞として独立したページに出力される。アブストラクトや本文と同じで、{\tt main.tex}に直接書いてもよいし、\verb|\include| コマンドを利用して{\tt include}してもよい。


\subsection{参考文献の出力}

\begin{itembox}[l]{{\tt main.tex}}
\begin{verbatim}
\include{91_bibliography}	% 参考文献。要独自コマンド、include先参照のこと
\end{verbatim}
\end{itembox}

謝辞に続いて、参考文献を出力する。

参考文献リストは、\verb|\begin{bib}| から \verb|\end{bib}| の間に、\verb|\bibitem| コマンドを使って書く。

BibTeXを使う場合は、以下のようにする。

\begin{itembox}[l]{{\tt 91\_bibliography.tex}}
\begin{verbatim}
\begin{bib}[100]
\bibliography{main}
\end{bib}
\end{verbatim}
\end{itembox}

こうすると、\verb|main.bib|から使用した参考文献のみを抽出して出力してくれる。\verb|main.bib|の中身は以下のようになっていて、気の利いた論文検索サイトであればBibTeXをコピペできるようになっているので簡単に作れるはず。


\begin{itembox}[l]{{\tt 91\_bibliography.tex}}
\begin{verbatim}
@article{hoge09,
    author  = "ほげ山太郎 and ほげ山次郎",
    yomi    = "ほげやまたろう",
    title   = "ほげほげ理論のHCI分野への応用",
    journal = "ほげほげ学会論文誌",
    volume  = "31",
    number  = "3",
    pages   = "194-201",
    year    = "2009",
}
@inproceedings{hoge08,
    author     = "Taro Hogeyama and Jiro Hogeyama",
    title      = "The Theory of Hoge",
    booktitle  = "The Proceedings of The Hoge Society",
    year       = "2008"
}
\end{verbatim}
\end{itembox}


以下は、BibTeXを使わないで手で書く例。

\begin{itembox}[l]{{\tt 91\_bibliography.tex}}
\begin{verbatim}
@article{hoge09,
    author  = "ほげ山太郎 and ほげ山次郎",
    yomi    = "ほげやまたろう",
    title   = "ほげほげ理論のHCI分野への応用",
    journal = "ほげほげ学会論文誌",
    volume  = "31",
    number  = "3",
    pages   = "194-201",
    year    = "2009",
}
@inproceedings{hoge08,
    author     = "Taro Hogeyama and Jiro Hogeyama",
    title      = "The Theory of Hoge",
    booktitle  = "The Proceedings of The Hoge Society",
    year       = "2008"
}
\end{verbatim}
\end{itembox}


英語の文献の場合、慣例的に書誌名をイタリック体にすることが多いらしい。

\begin{itembox}[l]{{\tt 91\_bibliography.tex}}
\begin{verbatim}
\begin{bib}[100]
\begin{thebibliography}{#1}
% \bibitem{参照用名称}
%   著者名: 
%   \newblock 文献名,
%   \newblock 書誌情報,出版年.

\bibitem{hoge09}
  ほげ山太郎,ほげ山次郎:
  \newblock ほげほげ理論のHCI分野への応用,
  \newblock ほげほげ学会論文誌,Vol.31,No.3,pp.194-201,2009.

\bibitem{hoge08}
  Taro Hogeyama, Jiro Hogeyama:
  \newblock The Theory of Hoge,
  \newblock {\it The Proceedings of The Hoge Society}, 2008.
\end{thebibliography}
\end{bib}
\end{verbatim}
\end{itembox}

\verb|\bibitem| コマンド中、参照用名称は、本文から参考文献を参照するときに使うので、忘れずに書いておく。参照文献を本文中に参照するときには、\verb|\cite{参照用名称}| のように書けばよい。例えば、この文の末尾には \verb|\cite{hoge09}| と書いてあるので、自動で対応する番号が振られる\cite{hoge09}\cite{hoge08}。

参考文献リストの番号付けと、本文で参照したときの番号の挿入は、全部が自動で行われる。ただしこれも、第\ref{sec:toc}節で説明した目次の出力と同じで、一時ファイルを生成してからの挿入なので、正しく出力するには最低でも二回のコンパイルが必要。BibTeXを使用する場合は、\verb|platex|コマンドのあと\verb|pbibtex|コマンドを実行し、さらに2回\verb|platex|コマンドを実行するといいらしい。



\subsection{付録の出力}

\begin{itembox}[l]{{\tt main.tex}}
\begin{verbatim}
\appendix
\include{92_appendix}		% 付録
\end{verbatim}
\end{itembox}

必要であれば、論文の最後には付録を出力する。

\verb|\appendix| コマンド以降に書いたものは、すべて付録として扱われる。付録部分の書き方は通常の本文とまったく同じで、\verb|\appendix| コマンド以降に書くだけで勝手に付録用の体裁で出力される。
	% 本文2
\chapter{目的と手法}
\label{chap:design}

この章では、本研究の目的を明らかにした後、どのような手法を用いるかを示す。

\section{目的}

本研究での目的は、ソースコードの理解に要するコストを下げ、デベロッパが既存のコードをより素早く容易に理解できるようになることである。

\section{手法}

ソースコードを理解することを難解にしている原因として、普段使っている自然言語との相違があることが考えられる。プログラミング言語は論理的な式の集合であることや、識別子などの記号的な表記がなされている。このように普段から使用している母国語との構造的な違いがプログラミング言語の理解を妨げている要因の一つである考えられる。よって、本論文ではプログラミング言語の一つであるpythonを例にとり、記述してある処理を母国語である日本語に変換し、ソースコードを理解しやすくなるのかを検証する。

\section{日本語変換}

プログラミング言語は記号を決められた文法通りに並べて処理を表現する。自然言語は様々な品詞の単語を並べて文とする。なのでプログラミング言語の記号的な表現をどうやって自然言語に置き換えるかが課題になる。
プログラミング言語の文法は抽象構文木という構造で定義される。pythonも例にもれず抽象構文木が定義されており、処理系によってチェックされる。pythonの抽象構文木は以下のように定義されている。

\begin{itembox}[l]{python抽象構文木}
\begin{verbatim}
module Python
{
    mod = Module(stmt* body)
        | Interactive(stmt* body)
        | Expression(expr body)

        -- not really an actual node but useful in Jython's typesystem.
        | Suite(stmt* body)

    stmt = FunctionDef(identifier name, arguments args,
                       stmt* body, expr* decorator_list, expr? returns)
          | AsyncFunctionDef(identifier name, arguments args,
                             stmt* body, expr* decorator_list, expr? returns)

          | ClassDef(identifier name,
             expr* bases,
             keyword* keywords,
             stmt* body,
             expr* decorator_list)
          | Return(expr? value)

          | Delete(expr* targets)
          | Assign(expr* targets, expr value)
          | AugAssign(expr target, operator op, expr value)

          -- use 'orelse' because else is a keyword in target languages
          | For(expr target, expr iter, stmt* body, stmt* orelse)
          | AsyncFor(expr target, expr iter, stmt* body, stmt* orelse)
          | While(expr test, stmt* body, stmt* orelse)
          | If(expr test, stmt* body, stmt* orelse)
          | With(withitem* items, stmt* body)
          | AsyncWith(withitem* items, stmt* body)

          | Raise(expr? exc, expr? cause)
          | Try(stmt* body, excepthandler* handlers, stmt* orelse, stmt* finalbody)
          | Assert(expr test, expr? msg)

          | Import(alias* names)
          | ImportFrom(identifier? module, alias* names, int? level)

          | Global(identifier* names)
          | Nonlocal(identifier* names)
          | Expr(expr value)
          | Pass | Break | Continue

          -- XXX Jython will be different
          -- col_offset is the byte offset in the utf8 string the parser uses
          attributes (int lineno, int col_offset)

          -- BoolOp() can use left & right?
    expr = BoolOp(boolop op, expr* values)
         | BinOp(expr left, operator op, expr right)
         | UnaryOp(unaryop op, expr operand)
         | Lambda(arguments args, expr body)
         | IfExp(expr test, expr body, expr orelse)
         | Dict(expr* keys, expr* values)
         | Set(expr* elts)
         | ListComp(expr elt, comprehension* generators)
         | SetComp(expr elt, comprehension* generators)
         | DictComp(expr key, expr value, comprehension* generators)
         | GeneratorExp(expr elt, comprehension* generators)
         -- the grammar constrains where yield expressions can occur
         | Await(expr value)
         | Yield(expr? value)
         | YieldFrom(expr value)
         -- need sequences for compare to distinguish between
         -- x < 4 < 3 and (x < 4) < 3
         | Compare(expr left, cmpop* ops, expr* comparators)
         | Call(expr func, expr* args, keyword* keywords)
         | Num(object n) -- a number as a PyObject.
         | Str(string s) -- need to specify raw, unicode, etc?
         | Bytes(bytes s)
         | NameConstant(singleton value)
         | Ellipsis

         -- the following expression can appear in assignment context
         | Attribute(expr value, identifier attr, expr_context ctx)
         | Subscript(expr value, slice slice, expr_context ctx)
         | Starred(expr value, expr_context ctx)
         | Name(identifier id, expr_context ctx)
         | List(expr* elts, expr_context ctx)
         | Tuple(expr* elts, expr_context ctx)

          -- col_offset is the byte offset in the utf8 string the parser uses
          attributes (int lineno, int col_offset)

    expr_context = Load | Store | Del | AugLoad | AugStore | Param

    slice = Slice(expr? lower, expr? upper, expr? step)
          | ExtSlice(slice* dims)
          | Index(expr value)

    boolop = And | Or

    operator = Add | Sub | Mult | MatMult | Div | Mod | Pow | LShift
                 | RShift | BitOr | BitXor | BitAnd | FloorDiv

    unaryop = Invert | Not | UAdd | USub

    cmpop = Eq | NotEq | Lt | LtE | Gt | GtE | Is | IsNot | In | NotIn

    comprehension = (expr target, expr iter, expr* ifs)

    excepthandler = ExceptHandler(expr? type, identifier? name, stmt* body)
                    attributes (int lineno, int col_offset)

    arguments = (arg* args, arg? vararg, arg* kwonlyargs, expr* kw_defaults,
                 arg? kwarg, expr* defaults)

    arg = (identifier arg, expr? annotation)
           attributes (int lineno, int col_offset)

    -- keyword arguments supplied to call (NULL identifier for **kwargs)
    keyword = (identifier? arg, expr value)

    -- import name with optional 'as' alias.
    alias = (identifier name, identifier? asname)

    withitem = (expr context_expr, expr? optional_vars)
}
\end{verbatim}
\end{itembox}

大まかには識別子などの記号の集合が式になり、さらに記号と式の集合が文になり、文を制御構造に倣って並べることによって処理を記述し、処理の集合をモジュール、つまり一つの翻訳単位と定義されている。先頭にAsyncがついている構文はバージョン3.5から追加された機能で、非同期処理を記述するためのものだが、本論文では考慮しないものとする。
本論文では、サブルーチンには考慮せず一つのメソッド内において日本語変換処理をする。

\section{複合文の変換}

pythonにおいて、複合文は制御構造を明示的に記述するために用いられる。複合文の処理の区切りはインデントで管理され、同じ深さのブロックは同じインデントで記述される。一つまたは文の集合として記述される。

\subsection{条件分岐}

条件分岐はtest式を条件式として、真の場合はbody文を、偽ならばorelse文を実行するという構成になっている。なので、文に変換するときは「targetの結果が成り立つならばbodyを処理、偽ならばorelseを処理」というように変換する。このように表記することでブロックの始まりと終わりを明示的に示す。

\subsection{ループ}

ループにはfor文かwhile文も用いる。for文の場合はiterで指定したイテレータの先頭からtargetに代入し、body文を処理し、すべての要素を代入し終えるとorelse文を処理するという構成になっている。pythonではループ回数を明示的に示すことはせずjavaの拡張for文のようになっている。なので、処理がわかるように日本語に変換すると「iterの最初の要素をtargetに代入して、body文を処理、iterの次の要素をtargetに代入してもう一度処理、最後の要素ならばorelse文を処理」というように記述できる。while文はtest式が真の限りbody文をループし、偽になるとorelse文を実行してループ処理を終了する。よってwhile文は「testに結果が成り立っている間、body文をループ。偽ならばorelse文を処理。」というように変換する。

\subsection{その他の複合文}

その他の複合文には、エラーを処理するtry~except~finally文と、処理を安全に記述するためのwith文がある。try文はbody文を処理しているときにhandler.typeで指定、または指定がなければ何らかのエラーが出た際にhandler.body文を処理する。orelse文は正常終了時に処理され、finally文はいかなる場合でも実行される。日本語に置き換えるには「body文の処理中に、handler.typeエラーが発生したら、handler.body文を実行	% 本文3
\include{04}	% 本文4
\end{verbatim}
\end{itembox}

目次に続いて、論文のメイン、本文を記述する。アブストラクトと同様で、{\tt main.tex}に直接書くか、\verb|\include| コマンドを利用して別に用意したファイルを{\tt include}する。

本文の書き方は、第\ref{chap:latex}章で詳しく説明する。


\subsection{謝辞の出力}

\begin{itembox}[l]{{\tt main.tex}}
\begin{verbatim}
\include{90_acknowledgment}	% 謝辞。要独自コマンド、include先参照のこと
\end{verbatim}
\end{itembox}

本文のあとには、謝辞を出力する。\verb|begin{acknowledgment}| から \verb|end{acknowledgment}| の間に書いた文章が、謝辞として独立したページに出力される。アブストラクトや本文と同じで、{\tt main.tex}に直接書いてもよいし、\verb|\include| コマンドを利用して{\tt include}してもよい。


\subsection{参考文献の出力}

\begin{itembox}[l]{{\tt main.tex}}
\begin{verbatim}
\include{91_bibliography}	% 参考文献。要独自コマンド、include先参照のこと
\end{verbatim}
\end{itembox}

謝辞に続いて、参考文献を出力する。

参考文献リストは、\verb|\begin{bib}| から \verb|\end{bib}| の間に、\verb|\bibitem| コマンドを使って書く。

BibTeXを使う場合は、以下のようにする。

\begin{itembox}[l]{{\tt 91\_bibliography.tex}}
\begin{verbatim}
\begin{bib}[100]
\bibliography{main}
\end{bib}
\end{verbatim}
\end{itembox}

こうすると、\verb|main.bib|から使用した参考文献のみを抽出して出力してくれる。\verb|main.bib|の中身は以下のようになっていて、気の利いた論文検索サイトであればBibTeXをコピペできるようになっているので簡単に作れるはず。


\begin{itembox}[l]{{\tt 91\_bibliography.tex}}
\begin{verbatim}
@article{hoge09,
    author  = "ほげ山太郎 and ほげ山次郎",
    yomi    = "ほげやまたろう",
    title   = "ほげほげ理論のHCI分野への応用",
    journal = "ほげほげ学会論文誌",
    volume  = "31",
    number  = "3",
    pages   = "194-201",
    year    = "2009",
}
@inproceedings{hoge08,
    author     = "Taro Hogeyama and Jiro Hogeyama",
    title      = "The Theory of Hoge",
    booktitle  = "The Proceedings of The Hoge Society",
    year       = "2008"
}
\end{verbatim}
\end{itembox}


以下は、BibTeXを使わないで手で書く例。

\begin{itembox}[l]{{\tt 91\_bibliography.tex}}
\begin{verbatim}
@article{hoge09,
    author  = "ほげ山太郎 and ほげ山次郎",
    yomi    = "ほげやまたろう",
    title   = "ほげほげ理論のHCI分野への応用",
    journal = "ほげほげ学会論文誌",
    volume  = "31",
    number  = "3",
    pages   = "194-201",
    year    = "2009",
}
@inproceedings{hoge08,
    author     = "Taro Hogeyama and Jiro Hogeyama",
    title      = "The Theory of Hoge",
    booktitle  = "The Proceedings of The Hoge Society",
    year       = "2008"
}
\end{verbatim}
\end{itembox}


英語の文献の場合、慣例的に書誌名をイタリック体にすることが多いらしい。

\begin{itembox}[l]{{\tt 91\_bibliography.tex}}
\begin{verbatim}
\begin{bib}[100]
\begin{thebibliography}{#1}
% \bibitem{参照用名称}
%   著者名: 
%   \newblock 文献名,
%   \newblock 書誌情報,出版年.

\bibitem{hoge09}
  ほげ山太郎,ほげ山次郎:
  \newblock ほげほげ理論のHCI分野への応用,
  \newblock ほげほげ学会論文誌,Vol.31,No.3,pp.194-201,2009.

\bibitem{hoge08}
  Taro Hogeyama, Jiro Hogeyama:
  \newblock The Theory of Hoge,
  \newblock {\it The Proceedings of The Hoge Society}, 2008.
\end{thebibliography}
\end{bib}
\end{verbatim}
\end{itembox}

\verb|\bibitem| コマンド中、参照用名称は、本文から参考文献を参照するときに使うので、忘れずに書いておく。参照文献を本文中に参照するときには、\verb|\cite{参照用名称}| のように書けばよい。例えば、この文の末尾には \verb|\cite{hoge09}| と書いてあるので、自動で対応する番号が振られる\cite{hoge09}\cite{hoge08}。

参考文献リストの番号付けと、本文で参照したときの番号の挿入は、全部が自動で行われる。ただしこれも、第\ref{sec:toc}節で説明した目次の出力と同じで、一時ファイルを生成してからの挿入なので、正しく出力するには最低でも二回のコンパイルが必要。BibTeXを使用する場合は、\verb|platex|コマンドのあと\verb|pbibtex|コマンドを実行し、さらに2回\verb|platex|コマンドを実行するといいらしい。



\subsection{付録の出力}

\begin{itembox}[l]{{\tt main.tex}}
\begin{verbatim}
\appendix
\include{92_appendix}		% 付録
\end{verbatim}
\end{itembox}

必要であれば、論文の最後には付録を出力する。

\verb|\appendix| コマンド以降に書いたものは、すべて付録として扱われる。付録部分の書き方は通常の本文とまったく同じで、\verb|\appendix| コマンド以降に書くだけで勝手に付録用の体裁で出力される。
	% 本文2
\chapter{目的と手法}
\label{chap:design}

この章では、本研究の目的を明らかにした後、どのような手法を用いるかを示す。

\section{目的}

本研究での目的は、ソースコードの理解に要するコストを下げ、デベロッパが既存のコードをより素早く容易に理解できるようになることである。

\section{手法}

ソースコードを理解することを難解にしている原因として、普段使っている自然言語との相違があることが考えられる。プログラミング言語は論理的な式の集合であることや、識別子などの記号的な表記がなされている。このように普段から使用している母国語との構造的な違いがプログラミング言語の理解を妨げている要因の一つである考えられる。よって、本論文ではプログラミング言語の一つであるpythonを例にとり、記述してある処理を母国語である日本語に変換し、ソースコードを理解しやすくなるのかを検証する。

\section{日本語変換}

プログラミング言語は記号を決められた文法通りに並べて処理を表現する。自然言語は様々な品詞の単語を並べて文とする。なのでプログラミング言語の記号的な表現をどうやって自然言語に置き換えるかが課題になる。
プログラミング言語の文法は抽象構文木という構造で定義される。pythonも例にもれず抽象構文木が定義されており、処理系によってチェックされる。pythonの抽象構文木は以下のように定義されている。

\begin{itembox}[l]{python抽象構文木}
\begin{verbatim}
module Python
{
    mod = Module(stmt* body)
        | Interactive(stmt* body)
        | Expression(expr body)

        -- not really an actual node but useful in Jython's typesystem.
        | Suite(stmt* body)

    stmt = FunctionDef(identifier name, arguments args,
                       stmt* body, expr* decorator_list, expr? returns)
          | AsyncFunctionDef(identifier name, arguments args,
                             stmt* body, expr* decorator_list, expr? returns)

          | ClassDef(identifier name,
             expr* bases,
             keyword* keywords,
             stmt* body,
             expr* decorator_list)
          | Return(expr? value)

          | Delete(expr* targets)
          | Assign(expr* targets, expr value)
          | AugAssign(expr target, operator op, expr value)

          -- use 'orelse' because else is a keyword in target languages
          | For(expr target, expr iter, stmt* body, stmt* orelse)
          | AsyncFor(expr target, expr iter, stmt* body, stmt* orelse)
          | While(expr test, stmt* body, stmt* orelse)
          | If(expr test, stmt* body, stmt* orelse)
          | With(withitem* items, stmt* body)
          | AsyncWith(withitem* items, stmt* body)

          | Raise(expr? exc, expr? cause)
          | Try(stmt* body, excepthandler* handlers, stmt* orelse, stmt* finalbody)
          | Assert(expr test, expr? msg)

          | Import(alias* names)
          | ImportFrom(identifier? module, alias* names, int? level)

          | Global(identifier* names)
          | Nonlocal(identifier* names)
          | Expr(expr value)
          | Pass | Break | Continue

          -- XXX Jython will be different
          -- col_offset is the byte offset in the utf8 string the parser uses
          attributes (int lineno, int col_offset)

          -- BoolOp() can use left & right?
    expr = BoolOp(boolop op, expr* values)
         | BinOp(expr left, operator op, expr right)
         | UnaryOp(unaryop op, expr operand)
         | Lambda(arguments args, expr body)
         | IfExp(expr test, expr body, expr orelse)
         | Dict(expr* keys, expr* values)
         | Set(expr* elts)
         | ListComp(expr elt, comprehension* generators)
         | SetComp(expr elt, comprehension* generators)
         | DictComp(expr key, expr value, comprehension* generators)
         | GeneratorExp(expr elt, comprehension* generators)
         -- the grammar constrains where yield expressions can occur
         | Await(expr value)
         | Yield(expr? value)
         | YieldFrom(expr value)
         -- need sequences for compare to distinguish between
         -- x < 4 < 3 and (x < 4) < 3
         | Compare(expr left, cmpop* ops, expr* comparators)
         | Call(expr func, expr* args, keyword* keywords)
         | Num(object n) -- a number as a PyObject.
         | Str(string s) -- need to specify raw, unicode, etc?
         | Bytes(bytes s)
         | NameConstant(singleton value)
         | Ellipsis

         -- the following expression can appear in assignment context
         | Attribute(expr value, identifier attr, expr_context ctx)
         | Subscript(expr value, slice slice, expr_context ctx)
         | Starred(expr value, expr_context ctx)
         | Name(identifier id, expr_context ctx)
         | List(expr* elts, expr_context ctx)
         | Tuple(expr* elts, expr_context ctx)

          -- col_offset is the byte offset in the utf8 string the parser uses
          attributes (int lineno, int col_offset)

    expr_context = Load | Store | Del | AugLoad | AugStore | Param

    slice = Slice(expr? lower, expr? upper, expr? step)
          | ExtSlice(slice* dims)
          | Index(expr value)

    boolop = And | Or

    operator = Add | Sub | Mult | MatMult | Div | Mod | Pow | LShift
                 | RShift | BitOr | BitXor | BitAnd | FloorDiv

    unaryop = Invert | Not | UAdd | USub

    cmpop = Eq | NotEq | Lt | LtE | Gt | GtE | Is | IsNot | In | NotIn

    comprehension = (expr target, expr iter, expr* ifs)

    excepthandler = ExceptHandler(expr? type, identifier? name, stmt* body)
                    attributes (int lineno, int col_offset)

    arguments = (arg* args, arg? vararg, arg* kwonlyargs, expr* kw_defaults,
                 arg? kwarg, expr* defaults)

    arg = (identifier arg, expr? annotation)
           attributes (int lineno, int col_offset)

    -- keyword arguments supplied to call (NULL identifier for **kwargs)
    keyword = (identifier? arg, expr value)

    -- import name with optional 'as' alias.
    alias = (identifier name, identifier? asname)

    withitem = (expr context_expr, expr? optional_vars)
}
\end{verbatim}
\end{itembox}

大まかには識別子などの記号の集合が式になり、さらに記号と式の集合が文になり、文を制御構造に倣って並べることによって処理を記述し、処理の集合をモジュール、つまり一つの翻訳単位と定義されている。先頭にAsyncがついている構文はバージョン3.5から追加された機能で、非同期処理を記述するためのものだが、本論文では考慮しないものとする。
本論文では、サブルーチンには考慮せず一つのメソッド内において日本語変換処理をする。

\section{複合文の変換}

pythonにおいて、複合文は制御構造を明示的に記述するために用いられる。複合文の処理の区切りはインデントで管理され、同じ深さのブロックは同じインデントで記述される。一つまたは文の集合として記述される。

\subsection{条件分岐}

条件分岐はtest式を条件式として、真の場合はbody文を、偽ならばorelse文を実行するという構成になっている。なので、文に変換するときは「targetの結果が成り立つならばbodyを処理、偽ならばorelseを処理」というように変換する。このように表記することでブロックの始まりと終わりを明示的に示す。

\subsection{ループ}

ループにはfor文かwhile文も用いる。for文の場合はiterで指定したイテレータの先頭からtargetに代入し、body文を処理し、すべての要素を代入し終えるとorelse文を処理するという構成になっている。pythonではループ回数を明示的に示すことはせずjavaの拡張for文のようになっている。なので、処理がわかるように日本語に変換すると「iterの最初の要素をtargetに代入して、body文を処理、iterの次の要素をtargetに代入してもう一度処理、最後の要素ならばorelse文を処理」というように記述できる。while文はtest式が真の限りbody文をループし、偽になるとorelse文を実行してループ処理を終了する。よってwhile文は「testに結果が成り立っている間、body文をループ。偽ならばorelse文を処理。」というように変換する。

\subsection{その他の複合文}

その他の複合文には、エラーを処理するtry~except~finally文と、処理を安全に記述するためのwith文がある。try文はbody文を処理しているときにhandler.typeで指定、または指定がなければ何らかのエラーが出た際にhandler.body文を処理する。orelse文は正常終了時に処理され、finally文はいかなる場合でも実行される。日本語に置き換えるには「body文の処理中に、handler.typeエラーが発生したら、handler.body文を実行	% 本文3
\include{04}	% 本文4
\end{verbatim}
\end{itembox}

目次に続いて、論文のメイン、本文を記述する。アブストラクトと同様で、{\tt main.tex}に直接書くか、\verb|\include| コマンドを利用して別に用意したファイルを{\tt include}する。

本文の書き方は、第\ref{chap:latex}章で詳しく説明する。


\subsection{謝辞の出力}

\begin{itembox}[l]{{\tt main.tex}}
\begin{verbatim}
\include{90_acknowledgment}	% 謝辞。要独自コマンド、include先参照のこと
\end{verbatim}
\end{itembox}

本文のあとには、謝辞を出力する。\verb|begin{acknowledgment}| から \verb|end{acknowledgment}| の間に書いた文章が、謝辞として独立したページに出力される。アブストラクトや本文と同じで、{\tt main.tex}に直接書いてもよいし、\verb|\include| コマンドを利用して{\tt include}してもよい。


\subsection{参考文献の出力}

\begin{itembox}[l]{{\tt main.tex}}
\begin{verbatim}
\include{91_bibliography}	% 参考文献。要独自コマンド、include先参照のこと
\end{verbatim}
\end{itembox}

謝辞に続いて、参考文献を出力する。

参考文献リストは、\verb|\begin{bib}| から \verb|\end{bib}| の間に、\verb|\bibitem| コマンドを使って書く。

BibTeXを使う場合は、以下のようにする。

\begin{itembox}[l]{{\tt 91\_bibliography.tex}}
\begin{verbatim}
\begin{bib}[100]
\bibliography{main}
\end{bib}
\end{verbatim}
\end{itembox}

こうすると、\verb|main.bib|から使用した参考文献のみを抽出して出力してくれる。\verb|main.bib|の中身は以下のようになっていて、気の利いた論文検索サイトであればBibTeXをコピペできるようになっているので簡単に作れるはず。


\begin{itembox}[l]{{\tt 91\_bibliography.tex}}
\begin{verbatim}
@article{hoge09,
    author  = "ほげ山太郎 and ほげ山次郎",
    yomi    = "ほげやまたろう",
    title   = "ほげほげ理論のHCI分野への応用",
    journal = "ほげほげ学会論文誌",
    volume  = "31",
    number  = "3",
    pages   = "194-201",
    year    = "2009",
}
@inproceedings{hoge08,
    author     = "Taro Hogeyama and Jiro Hogeyama",
    title      = "The Theory of Hoge",
    booktitle  = "The Proceedings of The Hoge Society",
    year       = "2008"
}
\end{verbatim}
\end{itembox}


以下は、BibTeXを使わないで手で書く例。

\begin{itembox}[l]{{\tt 91\_bibliography.tex}}
\begin{verbatim}
@article{hoge09,
    author  = "ほげ山太郎 and ほげ山次郎",
    yomi    = "ほげやまたろう",
    title   = "ほげほげ理論のHCI分野への応用",
    journal = "ほげほげ学会論文誌",
    volume  = "31",
    number  = "3",
    pages   = "194-201",
    year    = "2009",
}
@inproceedings{hoge08,
    author     = "Taro Hogeyama and Jiro Hogeyama",
    title      = "The Theory of Hoge",
    booktitle  = "The Proceedings of The Hoge Society",
    year       = "2008"
}
\end{verbatim}
\end{itembox}


英語の文献の場合、慣例的に書誌名をイタリック体にすることが多いらしい。

\begin{itembox}[l]{{\tt 91\_bibliography.tex}}
\begin{verbatim}
\begin{bib}[100]
\begin{thebibliography}{#1}
% \bibitem{参照用名称}
%   著者名: 
%   \newblock 文献名,
%   \newblock 書誌情報,出版年.

\bibitem{hoge09}
  ほげ山太郎,ほげ山次郎:
  \newblock ほげほげ理論のHCI分野への応用,
  \newblock ほげほげ学会論文誌,Vol.31,No.3,pp.194-201,2009.

\bibitem{hoge08}
  Taro Hogeyama, Jiro Hogeyama:
  \newblock The Theory of Hoge,
  \newblock {\it The Proceedings of The Hoge Society}, 2008.
\end{thebibliography}
\end{bib}
\end{verbatim}
\end{itembox}

\verb|\bibitem| コマンド中、参照用名称は、本文から参考文献を参照するときに使うので、忘れずに書いておく。参照文献を本文中に参照するときには、\verb|\cite{参照用名称}| のように書けばよい。例えば、この文の末尾には \verb|\cite{hoge09}| と書いてあるので、自動で対応する番号が振られる\cite{hoge09}\cite{hoge08}。

参考文献リストの番号付けと、本文で参照したときの番号の挿入は、全部が自動で行われる。ただしこれも、第\ref{sec:toc}節で説明した目次の出力と同じで、一時ファイルを生成してからの挿入なので、正しく出力するには最低でも二回のコンパイルが必要。BibTeXを使用する場合は、\verb|platex|コマンドのあと\verb|pbibtex|コマンドを実行し、さらに2回\verb|platex|コマンドを実行するといいらしい。



\subsection{付録の出力}

\begin{itembox}[l]{{\tt main.tex}}
\begin{verbatim}
\appendix
\include{92_appendix}		% 付録
\end{verbatim}
\end{itembox}

必要であれば、論文の最後には付録を出力する。

\verb|\appendix| コマンド以降に書いたものは、すべて付録として扱われる。付録部分の書き方は通常の本文とまったく同じで、\verb|\appendix| コマンド以降に書くだけで勝手に付録用の体裁で出力される。
	% 本文2
\chapter{目的と手法}
\label{chap:design}

この章では、本研究の目的を明らかにした後、どのような手法を用いるかを示す。

\section{目的}

本研究での目的は、ソースコードの理解に要するコストを下げ、デベロッパが既存のコードをより素早く容易に理解できるようになることである。

\section{手法}

ソースコードを理解することを難解にしている原因として、普段使っている自然言語との相違があることが考えられる。プログラミング言語は論理的な式の集合であることや、識別子などの記号的な表記がなされている。このように普段から使用している母国語との構造的な違いがプログラミング言語の理解を妨げている要因の一つである考えられる。よって、本論文ではプログラミング言語の一つであるpythonを例にとり、記述してある処理を母国語である日本語に変換し、ソースコードを理解しやすくなるのかを検証する。

\section{日本語変換}

プログラミング言語は記号を決められた文法通りに並べて処理を表現する。自然言語は様々な品詞の単語を並べて文とする。なのでプログラミング言語の記号的な表現をどうやって自然言語に置き換えるかが課題になる。
プログラミング言語の文法は抽象構文木という構造で定義される。pythonも例にもれず抽象構文木が定義されており、処理系によってチェックされる。pythonの抽象構文木は以下のように定義されている。

\begin{itembox}[l]{python抽象構文木}
\begin{verbatim}
module Python
{
    mod = Module(stmt* body)
        | Interactive(stmt* body)
        | Expression(expr body)

        -- not really an actual node but useful in Jython's typesystem.
        | Suite(stmt* body)

    stmt = FunctionDef(identifier name, arguments args,
                       stmt* body, expr* decorator_list, expr? returns)
          | AsyncFunctionDef(identifier name, arguments args,
                             stmt* body, expr* decorator_list, expr? returns)

          | ClassDef(identifier name,
             expr* bases,
             keyword* keywords,
             stmt* body,
             expr* decorator_list)
          | Return(expr? value)

          | Delete(expr* targets)
          | Assign(expr* targets, expr value)
          | AugAssign(expr target, operator op, expr value)

          -- use 'orelse' because else is a keyword in target languages
          | For(expr target, expr iter, stmt* body, stmt* orelse)
          | AsyncFor(expr target, expr iter, stmt* body, stmt* orelse)
          | While(expr test, stmt* body, stmt* orelse)
          | If(expr test, stmt* body, stmt* orelse)
          | With(withitem* items, stmt* body)
          | AsyncWith(withitem* items, stmt* body)

          | Raise(expr? exc, expr? cause)
          | Try(stmt* body, excepthandler* handlers, stmt* orelse, stmt* finalbody)
          | Assert(expr test, expr? msg)

          | Import(alias* names)
          | ImportFrom(identifier? module, alias* names, int? level)

          | Global(identifier* names)
          | Nonlocal(identifier* names)
          | Expr(expr value)
          | Pass | Break | Continue

          -- XXX Jython will be different
          -- col_offset is the byte offset in the utf8 string the parser uses
          attributes (int lineno, int col_offset)

          -- BoolOp() can use left & right?
    expr = BoolOp(boolop op, expr* values)
         | BinOp(expr left, operator op, expr right)
         | UnaryOp(unaryop op, expr operand)
         | Lambda(arguments args, expr body)
         | IfExp(expr test, expr body, expr orelse)
         | Dict(expr* keys, expr* values)
         | Set(expr* elts)
         | ListComp(expr elt, comprehension* generators)
         | SetComp(expr elt, comprehension* generators)
         | DictComp(expr key, expr value, comprehension* generators)
         | GeneratorExp(expr elt, comprehension* generators)
         -- the grammar constrains where yield expressions can occur
         | Await(expr value)
         | Yield(expr? value)
         | YieldFrom(expr value)
         -- need sequences for compare to distinguish between
         -- x < 4 < 3 and (x < 4) < 3
         | Compare(expr left, cmpop* ops, expr* comparators)
         | Call(expr func, expr* args, keyword* keywords)
         | Num(object n) -- a number as a PyObject.
         | Str(string s) -- need to specify raw, unicode, etc?
         | Bytes(bytes s)
         | NameConstant(singleton value)
         | Ellipsis

         -- the following expression can appear in assignment context
         | Attribute(expr value, identifier attr, expr_context ctx)
         | Subscript(expr value, slice slice, expr_context ctx)
         | Starred(expr value, expr_context ctx)
         | Name(identifier id, expr_context ctx)
         | List(expr* elts, expr_context ctx)
         | Tuple(expr* elts, expr_context ctx)

          -- col_offset is the byte offset in the utf8 string the parser uses
          attributes (int lineno, int col_offset)

    expr_context = Load | Store | Del | AugLoad | AugStore | Param

    slice = Slice(expr? lower, expr? upper, expr? step)
          | ExtSlice(slice* dims)
          | Index(expr value)

    boolop = And | Or

    operator = Add | Sub | Mult | MatMult | Div | Mod | Pow | LShift
                 | RShift | BitOr | BitXor | BitAnd | FloorDiv

    unaryop = Invert | Not | UAdd | USub

    cmpop = Eq | NotEq | Lt | LtE | Gt | GtE | Is | IsNot | In | NotIn

    comprehension = (expr target, expr iter, expr* ifs)

    excepthandler = ExceptHandler(expr? type, identifier? name, stmt* body)
                    attributes (int lineno, int col_offset)

    arguments = (arg* args, arg? vararg, arg* kwonlyargs, expr* kw_defaults,
                 arg? kwarg, expr* defaults)

    arg = (identifier arg, expr? annotation)
           attributes (int lineno, int col_offset)

    -- keyword arguments supplied to call (NULL identifier for **kwargs)
    keyword = (identifier? arg, expr value)

    -- import name with optional 'as' alias.
    alias = (identifier name, identifier? asname)

    withitem = (expr context_expr, expr? optional_vars)
}
\end{verbatim}
\end{itembox}

大まかには識別子などの記号の集合が式になり、さらに記号と式の集合が文になり、文を制御構造に倣って並べることによって処理を記述し、処理の集合をモジュール、つまり一つの翻訳単位と定義されている。先頭にAsyncがついている構文はバージョン3.5から追加された機能で、非同期処理を記述するためのものだが、本論文では考慮しないものとする。
本論文では、サブルーチンには考慮せず一つのメソッド内において日本語変換処理をする。

\section{複合文の変換}

pythonにおいて、複合文は制御構造を明示的に記述するために用いられる。複合文の処理の区切りはインデントで管理され、同じ深さのブロックは同じインデントで記述される。一つまたは文の集合として記述される。

\subsection{条件分岐}

条件分岐はtest式を条件式として、真の場合はbody文を、偽ならばorelse文を実行するという構成になっている。なので、文に変換するときは「targetの結果が成り立つならばbodyを処理、偽ならばorelseを処理」というように変換する。このように表記することでブロックの始まりと終わりを明示的に示す。

\subsection{ループ}

ループにはfor文かwhile文も用いる。for文の場合はiterで指定したイテレータの先頭からtargetに代入し、body文を処理し、すべての要素を代入し終えるとorelse文を処理するという構成になっている。pythonではループ回数を明示的に示すことはせずjavaの拡張for文のようになっている。なので、処理がわかるように日本語に変換すると「iterの最初の要素をtargetに代入して、body文を処理、iterの次の要素をtargetに代入してもう一度処理、最後の要素ならばorelse文を処理」というように記述できる。while文はtest式が真の限りbody文をループし、偽になるとorelse文を実行してループ処理を終了する。よってwhile文は「testに結果が成り立っている間、body文をループ。偽ならばorelse文を処理。」というように変換する。

\subsection{その他の複合文}

その他の複合文には、エラーを処理するtry~except~finally文と、処理を安全に記述するためのwith文がある。try文はbody文を処理しているときにhandler.typeで指定、または指定がなければ何らかのエラーが出た際にhandler.body文を処理する。orelse文は正常終了時に処理され、finally文はいかなる場合でも実行される。日本語に置き換えるには「body文の処理中に、handler.typeエラーが発生したら、handler.body文を実行	% 本文3
\include{04}	% 本文4
\end{verbatim}
\end{itembox}

目次に続いて、論文のメイン、本文を記述する。アブストラクトと同様で、{\tt main.tex}に直接書くか、\verb|\include| コマンドを利用して別に用意したファイルを{\tt include}する。

本文の書き方は、第\ref{chap:latex}章で詳しく説明する。


\subsection{謝辞の出力}

\begin{itembox}[l]{{\tt main.tex}}
\begin{verbatim}
\include{90_acknowledgment}	% 謝辞。要独自コマンド、include先参照のこと
\end{verbatim}
\end{itembox}

本文のあとには、謝辞を出力する。\verb|begin{acknowledgment}| から \verb|end{acknowledgment}| の間に書いた文章が、謝辞として独立したページに出力される。アブストラクトや本文と同じで、{\tt main.tex}に直接書いてもよいし、\verb|\include| コマンドを利用して{\tt include}してもよい。


\subsection{参考文献の出力}

\begin{itembox}[l]{{\tt main.tex}}
\begin{verbatim}
\include{91_bibliography}	% 参考文献。要独自コマンド、include先参照のこと
\end{verbatim}
\end{itembox}

謝辞に続いて、参考文献を出力する。

参考文献リストは、\verb|\begin{bib}| から \verb|\end{bib}| の間に、\verb|\bibitem| コマンドを使って書く。

BibTeXを使う場合は、以下のようにする。

\begin{itembox}[l]{{\tt 91\_bibliography.tex}}
\begin{verbatim}
\begin{bib}[100]
\bibliography{main}
\end{bib}
\end{verbatim}
\end{itembox}

こうすると、\verb|main.bib|から使用した参考文献のみを抽出して出力してくれる。\verb|main.bib|の中身は以下のようになっていて、気の利いた論文検索サイトであればBibTeXをコピペできるようになっているので簡単に作れるはず。


\begin{itembox}[l]{{\tt 91\_bibliography.tex}}
\begin{verbatim}
@article{hoge09,
    author  = "ほげ山太郎 and ほげ山次郎",
    yomi    = "ほげやまたろう",
    title   = "ほげほげ理論のHCI分野への応用",
    journal = "ほげほげ学会論文誌",
    volume  = "31",
    number  = "3",
    pages   = "194-201",
    year    = "2009",
}
@inproceedings{hoge08,
    author     = "Taro Hogeyama and Jiro Hogeyama",
    title      = "The Theory of Hoge",
    booktitle  = "The Proceedings of The Hoge Society",
    year       = "2008"
}
\end{verbatim}
\end{itembox}


以下は、BibTeXを使わないで手で書く例。

\begin{itembox}[l]{{\tt 91\_bibliography.tex}}
\begin{verbatim}
@article{hoge09,
    author  = "ほげ山太郎 and ほげ山次郎",
    yomi    = "ほげやまたろう",
    title   = "ほげほげ理論のHCI分野への応用",
    journal = "ほげほげ学会論文誌",
    volume  = "31",
    number  = "3",
    pages   = "194-201",
    year    = "2009",
}
@inproceedings{hoge08,
    author     = "Taro Hogeyama and Jiro Hogeyama",
    title      = "The Theory of Hoge",
    booktitle  = "The Proceedings of The Hoge Society",
    year       = "2008"
}
\end{verbatim}
\end{itembox}


英語の文献の場合、慣例的に書誌名をイタリック体にすることが多いらしい。

\begin{itembox}[l]{{\tt 91\_bibliography.tex}}
\begin{verbatim}
\begin{bib}[100]
\begin{thebibliography}{#1}
% \bibitem{参照用名称}
%   著者名: 
%   \newblock 文献名,
%   \newblock 書誌情報,出版年.

\bibitem{hoge09}
  ほげ山太郎,ほげ山次郎:
  \newblock ほげほげ理論のHCI分野への応用,
  \newblock ほげほげ学会論文誌,Vol.31,No.3,pp.194-201,2009.

\bibitem{hoge08}
  Taro Hogeyama, Jiro Hogeyama:
  \newblock The Theory of Hoge,
  \newblock {\it The Proceedings of The Hoge Society}, 2008.
\end{thebibliography}
\end{bib}
\end{verbatim}
\end{itembox}

\verb|\bibitem| コマンド中、参照用名称は、本文から参考文献を参照するときに使うので、忘れずに書いておく。参照文献を本文中に参照するときには、\verb|\cite{参照用名称}| のように書けばよい。例えば、この文の末尾には \verb|\cite{hoge09}| と書いてあるので、自動で対応する番号が振られる\cite{hoge09}\cite{hoge08}。

参考文献リストの番号付けと、本文で参照したときの番号の挿入は、全部が自動で行われる。ただしこれも、第\ref{sec:toc}節で説明した目次の出力と同じで、一時ファイルを生成してからの挿入なので、正しく出力するには最低でも二回のコンパイルが必要。BibTeXを使用する場合は、\verb|platex|コマンドのあと\verb|pbibtex|コマンドを実行し、さらに2回\verb|platex|コマンドを実行するといいらしい。



\subsection{付録の出力}

\begin{itembox}[l]{{\tt main.tex}}
\begin{verbatim}
\appendix
\include{92_appendix}		% 付録
\end{verbatim}
\end{itembox}

必要であれば、論文の最後には付録を出力する。

\verb|\appendix| コマンド以降に書いたものは、すべて付録として扱われる。付録部分の書き方は通常の本文とまったく同じで、\verb|\appendix| コマンド以降に書くだけで勝手に付録用の体裁で出力される。
