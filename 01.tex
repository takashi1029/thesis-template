\chapter{序論}
\label{chap:introduction}

この章では、まずこの研究を選んだ背景を述べた後、本論文の構成を示す。

\section{背景}

近年、IT化が進みソフトウェアの社会での必要性がますます強くなり、開発規模も大きくなってきており、ソフトウェアはチームで開発されることが多い。また昨今OSSの種類も増え、既存のコードを再利用する機会は開発において少なくない。

人がプログラムの内容を理解、または説明するときには二つの方法がある。それは実際にプログラムを動かしてそのログや動作や出力などから理解する動的解析と、ソースコードを読んで文の構造や処理を考えて理解する静的解析がある。動的解析は実際に動かすので、処理を追いやすく理解するコストは比較的低いと言えるが、出力結果やオブジェクトコードという形でしかプログラムを見ることができないので厳密な内部処理を理解することは難しい。また実際に動作させる上で環境構築するコストがかかってしまう。よって本論文では静的解析の補助を扱う。

\section{本文書の構成}

第\ref{chap:introduction}章では本研究の背景を書いた。第\ref{chap:related}章では、関連研究としてどのようなものがあるのかを示していく。第\ref{chap:design}章で研究の目的とどのような手法で問題を解決するのかを書き、第\ref{chap:implement}章では、具体的にどのように実装するかを示す。第\ref{chap:experiment}では評価方法と実際のアンケートの結果を分析する。第\ref{chap:conclusion}章で結論を示す。